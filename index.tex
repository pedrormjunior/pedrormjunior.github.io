\documentclass{article}
\usepackage[utf8x,utf8]{inputenc}
\usepackage{url}
\usepackage{hyperref}
\usepackage{enumitem}
\usepackage{ifthen}
\usepackage[acronym,nohypertypes={acronym}]{glossaries}
\usepackage{pifont}

% http://tex.stackexchange.com/a/245760/37199
\makeatletter
\newcommand{\newtabhref}[2]{\bgroup\let~\H@tilde%
  \Link[#1 target="_blank"]{}{}%
  {#2}\egroup\EndLink}%
\makeatother

\setlength\parindent{0pt}

\title{Pedro Ribeiro Mendes Júnior}
\author{\glsfirst{unicamp}\\%
  \glsfirst{ic}\\%
  \gls{recod}\\%
}
\date{\href{mailto:pedrormjunior@gmail.com}{pedrormjunior@gmail.com}}

\newcommand{\ifempty}[3]{\ifthenelse{\equal{#1}{}}{#2}{#3}}
\newcommand{\ifnotempty}[2]{\ifthenelse{\equal{#1}{}}{}{#2}}
\newcommand{\bibtex}[1]{\ifnotempty{#1}{\newtabhref{https://raw.githubusercontent.com/pedrormjunior/pedrormjunior.github.io/master/#1.bib}{[BibTeX]}}}
\newcommand{\myurl}[1]{\newtabhref{#1}{#1}}
\newcommand{\toppublication}{\ding{72}}
\newcommand{\formatdate}[3]{#2/#3/#1}

% #1: label
% #2: authors
% #3: year
% #4: doi
% #5: title
% #6: journal name
% #7: volume
% #8: issue
% #9: pages
% #10: BibTeX
\newcommand{\journal}[9]{\label{#1} #2 (#3), ``\ifempty{#4}{\underline{\textcolor{blue}{#5}}}{\newtabhref{#4}{#5}}'', \emph{\glsfirst{#6}}. \ifempty{#7}{\textbf{(\ifempty{#4}{to appear}{online})}}{#7\ifnotempty{#8}{(#8)}:#9.}\journalcontinued}
\newcommand{\journalcontinued}[1]{ \bibtex{#1}}

% #1: label
% #2: authors
% #3: year
% #4: doi
% #5: title
% #6: conference name
% #7: pages
% #8: BibTeX
\newcommand{\proceedings}[8]{\label{#1} #2 (#3), ``\ifempty{#4}{\underline{\textcolor{blue}{#5}}}{\newtabhref{#4}{#5}}'', In: \emph{\glsfirst{#6}}.\ifnotempty{#7}{ pp. #7.} \bibtex{#8}}

% #1: label
% #2: year
% #3: doi
% #4: title
% #5: institution
% #6: type
% #7: BibTeX
% #8: in Portuguese
\newcommand{\thesis}[8]{\label{#1} \gls{melast} (#2), ``\ifempty{#3}{\underline{\textcolor{blue}{#4}}}{\newtabhref{#3}{#4}}'' ({#6}\ifnotempty{#8}{; in Portuguese}), \emph{\glsfirst{#5}}. \bibtex{#7}}

% #1: label
% #2: authors
% #3: year
% #4: link
% #4: title
% #5: institution
% #6: type
% #7: number
% #8: filing date
% #9: publication date
\newcommand{\patent}[9]{\label{#1} #2 (#3), ``\ifempty{#4}{\underline{\textcolor{blue}{#5}}}{\newtabhref{#4}{#5}}'', \emph{\glsfirst{#6}}. \glsfirst{#7}: #8. Filing date: #9.\patentcontinue}
\newcommand{\patentcontinue}[1]{\ifnotempty{#1}{ Publication date: #1.}}

% #1: label
% #2: authors
% #3: year
% #4: link prefix
% #5: arXiv number
% #6: title
% #7: journal or conference
\newcommand{\arxiv}[7]{\label{#1} #2 (#3), ``\newtabhref{#4#5}{#6}'', \emph{arXiv:#5}. \ifempty{#7}{\textbf{(unpublished)}}{\textbf{(submitted to} \glsfirst{#7}\textbf{)}}}

\newcommand{\sepauthor}{\textcolor{blue}{\bf;}}
\newcommand{\authorname}[3][\sepauthor{}]{#3, #2#1}

% Journals
\newcommand{\definejournal}[2]{\newacronym[first={\glsentrytext{#1}}]{#1}{#2}{??}}
\definejournal{compbiomed}{Computers in Biology and Medicine}
\definejournal{jbhi}{Journal of Biomedical and Health Informatics}
\definejournal{ml}{Machine Learning}
\definejournal{jmlr}{Journal of Machine Learning Research}
\definejournal{fgcs}{Future Generation Computer Systems}
% Proceedings
\newacronym{wuwsibgrapi}{SIBGRAPI}{WUW Workshop of Undergraduate Work -- XXIII Conference on Graphics, Patterns and Image}
\newacronym{smc}{SMC}{IEEE Intl. Conference on Systems, Man, and Cybernetics}
\newacronym[first={\glsentrytext{mediaeval2014}}]{mediaeval2014}{MediaEval 2014 Workshop}{??}
\newacronym[first={\glsentrytext{mediaeval2015}}]{mediaeval2015}{MediaEval 2015 Workshop}{??}

% Institutions
\newacronym{unicamp}{UNICAMP}{University of Campinas}
\newacronym{ic}{IC}{Institute of Computing}
\newacronym[first={Reasoning for Complex Data (\glsentrytext{recod}) lab}]{recod}{RECOD}{??}
\newacronym{ufop}{UFOP}{Universidade Federal de Ouro Preto}
\newacronym{ufop-en}{UFOP}{Federal University of Ouro Preto}
\newacronym{uccs}{UCCS}{University of Colorado Colorado Springs}

% Patent institutions
\newacronym{inpi}{INPI}{Instituto Nacional da Propriedade Industrial}
\newacronym{uspto}{USPTO}{United States Patent and Trademark Office}

% Countries
\newacronym[first={\glsentrytext{br}}]{br}{Brazil}{??}
\newacronym[first={\glsentrytext{us}}]{us}{U.S. Patent}{??}

% Me
\newacronym[first={\glsentrytext{me}}]{me}{\textbf{\authorname{Pedro}{Mendes J{\'u}nior}}}{??}
\newacronym[first={\glsentrytext{melast}}]{melast}{\textbf{\authorname[]{Pedro}{Mendes J{\'u}nior}}}{??}

\begin{document}
\Css{
  body{
    margin-left: 3em;
    margin-right: 3em;
    margin-top: 3em;
    margin-bottom: 3em;
  }
}

\maketitle

I am a Ph.D. candidate at \gls{unicamp}, working mainly with recognition in open-set scenarios, investigating methods that can be applied to general open-set problems \ref{ref:MendesJunior2017} \ref{ref:MendesJunior2016a}.
I obtained the master's degree through \gls{unicamp} in open-set recognition \ref{ref:MendesJunior2014b}.
During the masters, I was member of a project financed by Samsung in which we have worked in medical problems as well \ref{ref:Pazinato2016} \ref{ref:Penatti2015}.
This project generated some patents for an open-set method \ref{ref:MendesJunior2014a} \ref{ref:MendesJunior2014} and for a heart view classification method \ref{ref:Rocha2014}.
Through \gls{ufop-en} I obtained the bachelor's degree in Computer Science \ref{ref:MendesJunior2012}.
During the undergraduate, I worked mainly with digital image processing in the specific problem of vehicle license plate location \ref{ref:MendesJunior2011} and I concluded my education studying the Haskell functional programming language for image processing \ref{ref:MendesJunior2012}.
I have worked at \gls{uccs} as a research assistant investigating open-set recognition for one year---2016--2017.

\newtabhref{http://lattes.cnpq.br/4535492803722330}{CV (in Portuguese)}
---
\newtabhref{https://scholar.google.com/citations?user=22PPnMAAAAAJ}{Google Scholar}
---
\newtabhref{https://recodbr.wordpress.com/team/students}{\gls{recod} page}
---
\newtabhref{https://github.com/pedrormjunior}{Github page}
---
\newtabhref{http://www.unicamp.br}{UNICAMP}
---
\newtabhref{https://www.ic.unicamp.br/en}{IC}
---
\newtabhref{http://www.recod.ic.unicamp.br}{RECOD}

\section*{Publications}

\subsection*{Journal articles}

\begin{enumerate}[label={[\arabic*]}]
\item\journal{ref:Werneck2018}
  {
    \authorname{Rafael}{Werneck}
    \authorname{Waldir}{de Almeida}
    \authorname{Bernardo}{Stein}
    \authorname{Daniel}{Pazinato}
    \gls{me}
    \authorname{Otávio}{Penatti}
    \authorname[]{Anderson}{Rocha}
    \authorname{Ricardo}{Torres}
  }
  {2018}
  {http://www.sciencedirect.com/science/article/pii/S0167739X17301565}
  {Kuaa: A unified framework for design, deployment, execution, and recommendation of machine learning experiments}
  {fgcs}
  {78}
  {1}
  {59--76}
  {Werneck2018}
\item\journal{ref:MendesJunior2017}
  {
    \gls{me}
    \authorname{Roberto}{Souza}
    \authorname{Rafael}{Werneck}
    \authorname{Bernardo}{Stein}
    \authorname{Daniel}{Pazinato}
    \authorname{Waldir}{de Almeida}
    \authorname{Otávio}{Penatti}
    \authorname{Ricardo}{Torres}
    \authorname[]{Anderson}{Rocha}
  }
  {2017}
  {http://link.springer.com/article/10.1007/s10994-016-5610-8}
  {Nearest neighbors distance ratio open-set classifier}
  {ml}
  {106}
  {3}
  {359--386}
  {MendesJunior2017}
\item\journal{ref:Pazinato2016}
  {
    \authorname{Daniel}{Pazinato}
    \authorname{Bernardo}{Stein}
    \authorname{Waldir}{de Almeida}
    \authorname{Rafael}{Werneck}
    \gls{me}
    \authorname{Otávio}{Penatti}
    \authorname{Ricardo}{Torres}
    \authorname{Fabio}{Menezes}
    \authorname[]{Anderson}{Rocha}
  }
  {2016}
  {http://dx.doi.org/10.1109/JBHI.2014.2386796}
  {Pixel-Level Tissue Classification for Ultrasound Images}
  {jbhi}
  {20}
  {1}
  {256--267}
  {Pazinato2016}
\item\journal{ref:Penatti2015}
  {
    \authorname{Otávio}{Penatti}
    \authorname{Rafael}{Werneck}
    \authorname{Waldir}{de Almeida}
    \authorname{Bernardo}{Stein}
    \authorname{Daniel}{Pazinato}
    \gls{me}
    \authorname{Ricardo}{Torres}
    \authorname[]{Anderson}{Rocha}
  }
  {2015}
  {http://dx.doi.org/10.1016/j.compbiomed.2015.08.004}
  {Mid-level Image Representations for Real-time Heart View Plane Classification of Echocardiograms}
  {compbiomed}
  {66}
  {}
  {66--81}
  {Penatti2015}
\end{enumerate}

\subsection*{Proceedings}

\begin{enumerate}[label={[\arabic*]},resume]
\item\proceedings{ref:Li2015}
  {
    \authorname{Lin}{Li}
    \authorname{Javier}{Muñoz}
    \authorname{Jurandy}{Almeida}
    \authorname{Rodrigo}{Calumby}
    \authorname{Otávio}{Penatti}
    \authorname{Ícaro}{Dourado}
    \authorname{Keiller}{Nogueira}
    \gls{me}
    \authorname{Luís}{Pereira}
    \authorname{Daniel}{Pedronette}
    \authorname{Jefersson}{dos Santos}
    \authorname{Marcos}{Gonçalvez}
    \authorname[]{Ricardo}{Torres}
  }
  {2015}
  {http://ceur-ws.org/Vol-1436/Paper49.pdf}
  {RECOD @ Placing Task of MediaEval 2015}
  {mediaeval2015}
  {1--3}
  {}
\item\proceedings{ref:Li2014}
  {
    \authorname{Lin}{Li}
    \authorname{Otávio}{Penatti}
    \authorname{Jurandy}{Almeida}
    \authorname{Giovani}{Chiachia}
    \authorname{Rodrigo}{Calumby}
    \gls{me}
    \authorname{Daniel}{Pedronette}
    \authorname[]{Ricardo}{Torres}
  }
  {2014}
  {http://ceur-ws.org/Vol-1263/mediaeval2014_submission_81.pdf}
  {Multimedia geocoding: The RECOD 2014 approach}
  {mediaeval2014}
  {1--2}
  {}
\item\proceedings{ref:MendesJunior2011}
  {
    \gls{me}
    \authorname{José Maria}{Neves}
    \authorname{Andrea}{Taveres}
    \authorname[]{David}{Menotti}
  }
  {2011}
  {http://dx.doi.org/10.1109/icsmc.2011.6084108}
  {Towards an automatic vehicle access control system: License plate location}
  {smc}
  {2916--2921}
  {MendesJunior2011}
\item\proceedings{ref:MendesJunior2010}
  {
    \gls{me}
    \authorname{José Maria}{Neves}
    \authorname[]{David}{Menotti}
  }
  {2010}
  {}
  {Em rumo a um sistema automático de controle de acesso de veículos automotivos: Localização de placas de identificação}
  {wuwsibgrapi}
  {}
  {}
\end{enumerate}

\subsection*{Submitted papers}

\begin{enumerate}[label={[\arabic*]},resume]
\item\arxiv{ref:MendesJunior2016a}
  {
    \gls{me}
    \authorname{Terrance}{Boult}
    \authorname{Jacques}{Wainer}
    \authorname[]{Anderson}{Rocha}
  }
  {2016}
  {https://arxiv.org/abs/}
  {1606.03802}
  {Specialized Support Vector Machines for open-set recognition}
  {}
\end{enumerate}

\subsection*{Theses}

% #1: label
% #2: year
% #3: doi
% #4: title
% #5: institution
% #6: BibTeX
\begin{enumerate}[label={[\arabic*]},resume]
\item\thesis{ref:MendesJunior2014b}
  {2014}
  {http://www.bibliotecadigital.unicamp.br/document/?code=000935853}
  {Open-set optimum-path forest classifier}
  {unicamp}
  {MSc}
  {}
  {}
\item\thesis{ref:MendesJunior2012}
  {2012}
  {http://www.decom.ufop.br/menotti/monoII121/files/BCC391-121-vf-08.2.4114-PedroRibeiroMendesJunior.pdf}
  {Uso de paralelismo de dados em algoritmos de processamento de imagens utilizando Haskell}
  {ufop}
  {BCS}
  {}
  {true}
\end{enumerate}

\subsection*{Patents}

\begin{enumerate}[label={[\arabic*]},resume]
\item\patent{ref:MendesJunior2014a}
  {
    \gls{me}
    \authorname{Roberto}{Souza}
    \authorname{Rafael}{Werneck}
    \authorname{Bernardo}{Stein}
    \authorname{Daniel}{Pazinato}
    \authorname{Waldir}{de Almeida}
    \authorname{Otávio}{Penatti}
    \authorname{Ricardo}{Torres}
    \authorname[]{Anderson}{Rocha}
  }
  {2014}
  {https://patents.google.com/patent/US20160092790A1}
  {Method for Multiclass Classification in Open-set Scenarios and uses thereof}
  {uspto}
  {us}
  {14532580}
  {\formatdate{2014}{11}{04}}
  {\formatdate{2016}{03}{31}}
\item\patent{ref:MendesJunior2014}
  {
    \gls{me}
    \authorname{Roberto}{Souza}
    \authorname{Rafael}{Werneck}
    \authorname{Bernardo}{Stein}
    \authorname{Daniel}{Pazinato}
    \authorname{Waldir}{de Almeida}
    \authorname{Otávio}{Penatti}
    \authorname{Ricardo}{Torres}
    \authorname[]{Anderson}{Rocha}
  }
  {2014}
  {https://worldwide.espacenet.com/publicationDetails/originalDocument?FT=D&date=20160517&DB=&locale=&CC=BR&NR=102014023780A2&KC=A2&ND=1}
  {Método para classificação multiclasse em cenários abertos e usos do mesmo}
  {inpi}
  {br}
  {BR10201402378}
  {\formatdate{2014}{09}{25}}
  {\formatdate{2016}{05}{17}}
\item\patent{ref:Rocha2014}
  {
    \authorname{Anderson}{Rocha}
    \authorname{Ricardo}{Torres}
    \authorname{Otávio}{Penatti}
    \gls{me}
    \authorname{Daniel}{Pazinato}
    \authorname{Waldir}{de Almeida}
    \authorname{Rafael}{Werneck}
    \authorname[]{Bernardo}{Stein}
  }
  {2014}
  {}
  {Método para classificação automática de visões do coração a partir de ecocardiogramas}
  {inpi}
  {br}
  {BR10201401105}
  {\formatdate{2014}{05}{07}}
  {}
\end{enumerate}

\section*{Reproducibility}
\label{sec:reproducibility}

Feature vectors used in \ref{ref:MendesJunior2017}: \myurl{https://dx.doi.org/10.6084/m9.figshare.1097614}.

Results and statistical analysis of \ref{ref:MendesJunior2016a}: \myurl{https://github.com/pedrormjunior/ssvm-results}.

Proposed algorithm and baselines used in \ref{ref:MendesJunior2011}: \myurl{https://github.com/pedrormjunior/vlpl}.

Datasets used in \ref{ref:MendesJunior2011}: \myurl{https://github.com/pedrormjunior/vlpl/blob/master/imgs-database}.

\section*{News}

\begin{enumerate}[label={[\arabic*]},resume]
  \item \newtabhref{https://recodbr.wordpress.com/2016/06/02/recod-is-among-the-winners-of-unicamp-inventor-award-2016}{RECOD is among the winners of Unicamp Inventor Award 2016}
\end{enumerate}

\begin{flushright}
  \footnotesize Last \hfill updated on \today{}.
\end{flushright}

\end{document}
