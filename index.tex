% -*- eval: (git-gutter-mode); -*-
% -*- eval: (git-gutter-mode); -*-
\documentclass{article}
\usepackage[utf8x,utf8]{inputenc}
\usepackage{url}
\usepackage[
colorlinks,
urlcolor={blue},
linkcolor={blue},
]{hyperref}
\usepackage{enumitem}
\usepackage{ifthen}
\usepackage[acronym,nohypertypes={acronym}]{glossaries}
\usepackage{pifont}
\usepackage{xr-hyper} %http://compgroups.net/comp.text.tex/using-tex4ht-with-hyperref-xr/1913453
\usepackage{color}
\usepackage{tabularx}

% https://tex.stackexchange.com/a/142973
\usepackage[USenglish]{babel}
\usepackage[useregional]{datetime2}
\newcommand{\formatdate}[3]{\DTMdisplaydate{#1}{#2}{#3}{-1}}

\newcounter{publicationcounter}
\newcommand{\pubref}[1]{[#1]}
\newcommand{\pub}[1]{\pubref{\ref{#1}}}
\newcommand{\pubs}[2]{\pubref{\ref{#1}, \ref{#2}}}
\newcommand{\pubitem}[1]{\refstepcounter{publicationcounter}\label{#1}\pubref{\thepublicationcounter} & }

\newenvironment{footenv}{
  \vspace*{1cm}                 %affects only PDFs
  \begin{footnotesize}
  }{
    {\footnotesize Last updated on\\\today{}}
  \end{footnotesize}
}
\newenvironment{footinfo}{}{}

% https://tex.stackexchange.com/a/42331
\newenvironment{publications}{
  \tabularx{\textwidth}{rX}
}{
  \endtabularx
}

\newcommand{\newtabhref}[2]{\href{#1}{#2}}

\newcommand{\blackhref}[3][black]{%
  \href{#2}{\textcolor{#1}{#3}}%
}

\setlength\parindent{0pt}
\setlength\fboxrule{1pt}
\setlength\fboxsep{2pt}

\newcommand{\ifempty}[3]{\ifthenelse{\equal{#1}{}}{#2}{#3}}
\newcommand{\ifnotempty}[2]{\ifthenelse{\equal{#1}{}}{}{#2}}
\newcommand{\speciallink}[2]{\href{#1}{\fbox{#2}}}
\newcommand{\bibtex}[1]{\ifnotempty{#1}{\speciallink{#1.bib.html}{BibTeX}}}
\newcommand{\myurl}[1]{\newtabhref{#1}{#1}}
\newcommand{\toppublication}{\ding{72}}

% Tags
\newenvironment{tags}{}{}
\newcommand{\definetag}[3]{%
  % #1: tag key
  % #2: link
  % #3: tag
  \newacronym[first={\glsentrytext{#1}}]{#1}{\newtabhref{#2}{\framebox{#3}}}{??}
}
\definetag{Lattes}{http://lattes.cnpq.br/4535492803722330}{CV~(Portuguese)}
\definetag{GoogleScholar}{https://scholar.google.com/citations?user=22PPnMAAAAAJ}{Google~Scholar}
\definetag{ResearchGate}{https://www.researchgate.net/profile/Pedro_Junior16}{ResearchGate}
\definetag{RECODpage}{https://recodbr.wordpress.com/team/students}{\gls{recod}~page}
\definetag{Github}{https://github.com/pedrormjunior}{Github}
\definetag{CV}{http://pedrormjunior.github.io/CV.pdf}{CV~(PDF)}
\definetag{Publons}{https://publons.com/researcher/T-3501-2019/}{Publons}
\definetag{ORCID}{https://orcid.org/0000-0001-8086-018X}{ORCID}

% #1: label
% #2: authors
% #3: year
% #4: doi
% #5: title
% #6: journal name
% #7: volume
% #8: issue
% #9: pages
% #10: BibTeX
\newcommand{\journal}[9]{\pubitem{#1} #2 (#3), ``\ifempty{#4}{\textcolor{blue}{#5}}{\newtabhref{#4}{#5}}'', \emph{\glsfirst{#6}}. \ifempty{#7}{\textbf{(\ifempty{#4}{to appear}{online})}}{#7\ifnotempty{#8}{(#8)}:#9.}\journalcontinued}
\newcommand{\journalcontinued}[1]{ \bibtex{#1}}

% #1: label
% #2: authors
% #3: year
% #4: doi
% #5: title
% #6: conference name
% #7: pages
% #8: to appear (not empty)
% #9: BibTeX
\newcommand{\proceedings}[9]{\pubitem{#1} #2 (#3), ``\ifempty{#4}{\textcolor{blue}{#5}}{\newtabhref{#4}{#5}}'', In: \emph{\glsfirst{#6}}.\ifnotempty{#7}{ pp. #7.}\ifnotempty{#8}{ \textbf{(to appear)}}{} \bibtex{#9}}

% #1: label
% #2: year
% #3: doi
% #4: title
% #5: institution
% #6: type
% #7: BibTeX
% #8: in Portuguese
\newcommand{\thesis}[8]{\pubitem{#1} \authorsymbol[]{PedroMendesJunior} (#2), ``\ifempty{#3}{\textcolor{blue}{#4}}{\newtabhref{#3}{#4}}'' ({#6}\ifnotempty{#8}{; in Portuguese}), \emph{\glsfirst{#5}}. \bibtex{#7}}

% #1: label
% #2: authors
% #3: year
% #4: link
% #5: title
% #6: institution
% #7: type (ignoring as of 20190610)
% #8: number
% #9: filing date
% #10: publication date
% #11: issue date
\newcommand{\patent}[9]{\pubitem{#1} #2 (#3), ``\ifempty{#4}{\textcolor{blue}{#5}}{\newtabhref{#4}{#5}}'', \emph{\glsfirst{#6}}. #8. Filing date: #9.\patentcontinue}
\newcommand{\patentcontinue}[2]{\ifnotempty{#1}{ Publication date: #1.}\ifnotempty{#2}{ Issue date: #2.}}

% #1: label
% #2: authors
% #3: year
% #4: arXiv number
% #5: title
% #6: journal or conference
% #7: BibTeX
\newcommand{\arxiv}[7]{\pubitem{#1} #2 (#3), ``\newtabhref{https://arxiv.org/abs/#4}{#5}'', \emph{arXiv:#4}. \ifempty{#6}{\textbf{(unpublished)}}{\textbf{(to appear in} \glsfirst{#6}\textbf{)}} \bibtex{#7}}

\newcommand{\sepauthor}{\textcolor{blue}{\bf;}}
\newcommand{\authorname}[3][\sepauthor{}]{#3, #2#1}
\newcommand{\authorsymbol}[2][\sepauthor{}]{\glssymbol{#2}#1}
\newcommand{\authorref}[2]{\hyperref[#2]{#1~\pub{#2}}}

% Journals
\newcommand{\definejournal}[3]{%
  % #1: journal key
  % #2: website
  % #3: journal name
  \newacronym[first={\glsentrytext{#1}}]{#1}{\blackhref{#2}{#3}}{??}
}
\definejournal{compbiomed}{https://www.journals.elsevier.com/computers-in-biology-and-medicine/}{Elsevier Computers in Biology and Medicine}
\definejournal{jbhi}{https://jbhi.embs.org/}{IEEE Journal of Biomedical and Health Informatics}
\definejournal{ml}{https://www.springer.com/journal/10994}{Springer Machine Learning}
\definejournal{fgcs}{https://www.journals.elsevier.com/future-generation-computer-systems/}{Elsevier Future Generation Computer Systems}
\definejournal{a}{https://ieeeaccess.ieee.org/}{IEEE Access}
\definejournal{TITS}{https://ieeexplore.ieee.org/xpl/RecentIssue.jsp?punumber=6979}{IEEE Transactions on Intelligent Transportation Systems}
\definejournal{SPL}{https://ieeexplore.ieee.org/xpl/RecentIssue.jsp?punumber=97}{IEEE Signal Processing Letters}
\definejournal{TSMCS}{https://ieeexplore.ieee.org/xpl/RecentIssue.jsp?punumber=6221021}{IEEE Transactions on Systems, Man, and Cybernetics: Systems}


% Proceedings
\newcommand{\defineconference}[4]{%
  % #1: conference key
  % #2: website
  % #3: conference abbreviation
  % #4: conference name or ?? to ignore it
  \ifthenelse{\equal{#4}{??}}{
    \newacronym[first={\blackhref{#2}{\glsentrytext{#1}}}]{#1}{#3}{#4}
  }{
    \newacronym[first={\blackhref{#2}{\glsentrydesc{#1} (\glsentrytext{#1})}}]{#1}{#3}{#4}
  }
}
\defineconference{wuwsibgrapi2010}{http://www.inf.ufrgs.br/sibgrapi2010/}{SIBGRAPI}{Workshop of Undergraduate Work, Conference on Graphics, Patterns and Image}
\defineconference{smc2011}{http://www.smc2011.org/}{SMC}{IEEE International Conference on Systems, Man, and Cybernetics}
\defineconference{mediaeval2014}{http://www.multimediaeval.org/mediaeval2014/}{MediaEval 2014 Workshop}{??}
\defineconference{mediaeval2015}{http://www.multimediaeval.org/mediaeval2015/}{MediaEval 2015 Workshop}{??}
\defineconference{icip2019}{http://www.2019.ieeeicip.org/}{ICIP}{IEEE International Conference on Image Processing}

% Institutions
\newacronym{unicamp}{UNICAMP}{University of Campinas}
\newacronym{ic}{IC}{Institute of Computing}
\newacronym[first={Reasoning for Complex Data (\glsentrytext{recod}) lab}]{recod}{RECOD}{??}
\newacronym{ufop}{UFOP}{Universidade Federal de Ouro Preto}
\newacronym{ufop-en}{UFOP}{Federal University of Ouro Preto}
\newacronym{uccs}{UCCS}{University of Colorado Colorado Springs}
\newacronym[first={\glsentrytext{polimi} (\glsentrydesc{polimi})}]{polimi}{Politecnico di Milano}{Polimi}
\newacronym{deib}{DEIB}{Dipartimento di Elettronica, Informazione e Bioingegneria}
\newacronym{ispg}{ISPG}{Image and Sound Processing Group}

% People
% \glsfirst: full name with hyperlink
% \glsdesc: full name
% \glstext: first name
% \glssymbol: author name (in inverse way)
\newcommand\highlightauthorname[1]{\textbf{#1}}
\newcommand\newauthor[5]{%
  % #1: author key
  % #2: website
  % #3: first name
  % #4: full name
  % #5: author name (in inverse way)
  \newacronym[first={\href{#2}{\glsentrydesc{#1}}},symbol={\blackhref{#2}{#5}}]{#1}{#3}{#4}
}
\newauthor{PedroMendesJunior}{https://pedrormjunior.github.io/}{Pedro}{Pedro Ribeiro Mendes Júnior}{\highlightauthorname{Mendes Júnior, Pedro}}
\newauthor{DavidMenotti}{https://web.inf.ufpr.br/menotti/}{David}{David Menotti Gomes}{Menotti, David}
\newauthor{JoseMariaNeves}{http://www.decom.ufop.br/jmneves/}{José Maria}{José Maria Ribeiro Neves}{Neves, José Maria}
\newauthor{AndreaTavares}{http://lattes.cnpq.br/7851263252868872}{Andrea}{Andrea Iabrudi Tavares}{Tavares, Andrea}
\newauthor{RicardoTorres}{http://www.ic.unicamp.br/~rtorres/}{Ricardo}{Ricardo da Silva Torres}{Torres, Ricardo}
\newauthor{AndersonRocha}{http://www.ic.unicamp.br/~rocha/}{Anderson}{Anderson de Rezende Rocha}{Rocha, Anderson}
\newauthor{ManuelNeira}{https://orcid.org/0000-0002-6527-6740}{Manuel}{Manuel Alberto Córdova Neira}{Neira, Manuel}
\newauthor{RafaelWerneck}{http://www.recod.ic.unicamp.br/~rwerneck/}{Rafael}{Rafael de Oliveira Werneck}{Werneck, Rafael}
\newauthor{OtavioPenatti}{http://www.recod.ic.unicamp.br/~otavio/academico/index.htm}{Otávio}{Otávio Augusto Bizetto Penatti}{Penatti, Otávio}
\newauthor{WaldirAlmeida}{http://lattes.cnpq.br/9612173318358087}{Waldir}{Waldir Rodrigues de Almeida}{de Almeida, Waldir}
\newauthor{DanielPazinato}{http://lattes.cnpq.br/0870783723224694}{Daniel}{Daniel Vatanabe Pazinato}{Pazinato, Daniel}
\newauthor{BernardoStein}{http://lattes.cnpq.br/9651478878569720}{Bernardo}{Bernardo Vecchia Stein}{Stein, Bernardo}
\newauthor{RobertoSouza}{http://lattes.cnpq.br/0692029921953122}{Roberto}{Roberto Medeiros de Souza}{de Souza, Roberto}
\newauthor{FabioMenezes}{http://lattes.cnpq.br/6434135168928035}{Fábio}{Fábio Hüsemann Menezes}{Menezes, Fábio}
\newauthor{LinLi}{http://lattes.cnpq.br/0781488945777221}{Lin}{Lin Tzy Li}{Li, Lin}
\newauthor{JavierMunoz}{http://lattes.cnpq.br/5385182122254592}{Javier}{Javier Muñoz}{Muñoz, Javier}
\newauthor{JurandyAlmeida}{http://lattes.cnpq.br/4495269939725770}{Jurandy}{Jurandy Gomes de Almeida Junior}{Almeida, Jurandy}
\newauthor{RodrigoCalumby}{http://lattes.cnpq.br/3303713473565543}{Rodrigo}{Rodrigo Tripodi Calumby}{Calumby, Rodrigo}
\newauthor{IcaroDourado}{http://lattes.cnpq.br/1896476549860669}{Ícaro}{Ícaro Cavalcante Dourado}{Dourado, Ícaro}
\newauthor{KeillerNogueira}{http://lattes.cnpq.br/1907957530680336}{Keiller}{Keiller Nogueira}{Nogueira, Keiller}
\newauthor{LuisPereira}{http://lattes.cnpq.br/5228991166855582}{Luís}{Luis Augusto Martins Pereira}{Pereira, Luís}
\newauthor{DanielPedronette}{http://lattes.cnpq.br/3363615000303340}{Daniel}{Daniel Carlos Guimarães Pedronette}{Pedronette, Daniel}
\newauthor{GiovaniChiachia}{http://lattes.cnpq.br/0507994043045945}{Giovani}{Giovani Chiachia}{Chiachia, Giovani}
\newauthor{TerranceBoult}{https://www.uccs.edu/cs/about/faculty/terrance-boult}{Terrance}{Terrance Edward Boult}{Boult, Terrance}
\newauthor{JacquesWainer}{http://www.ic.unicamp.br/~wainer/}{Jacques}{Jacques Wainer}{Wainer, Jacques}
\newauthor{JeferssonSantos}{https://homepages.dcc.ufmg.br/~jefersson/}{Jefersson}{Jefersson Alex dos Santos}{dos Santos, Jefersson}
\newauthor{MarcosGoncalves}{https://www.dcc.ufmg.br/dcc/?q=pt-br/node/226}{Marcos}{Marcos André Gonçalves}{Gonçalves, Marcos}
\newauthor{LucaBondi}{http://home.deib.polimi.it/lbondi/index.html}{Luca}{Luca Bondi}{Bondi, Luca}
\newauthor{PaoloBestagini}{http://home.deib.polimi.it/bestagini/}{Paolo}{Paolo Bestagini}{Bestagini, Paolo}
\newauthor{StefanoTubaro}{http://home.deib.polimi.it/tubaro/}{Stefano}{Stefano Tubaro}{Tubaro, Stefano}

% Patent institutions
\newacronym{inpi}{INPI}{Instituto Nacional da Propriedade Industrial}
\newacronym{uspto}{USPTO}{United States Patent and Trademark Office}

% Countries
\newacronym[first={\glsentrytext{br}}]{br}{Brazil}{??}
\newacronym[first={\glsentrytext{us}}]{us}{U.S. Patent}{??}

% Proceedings general entries.
\newcommand\MendesJunior[1]{%
  \ifthenelse{\equal{#1}{2016}}{%
    \arxiv{ref:MendesJunior2016a}
    {
      \authorsymbol{PedroMendesJunior}
      \authorsymbol{TerranceBoult}
      \authorsymbol{JacquesWainer}
      \authorsymbol[]{AndersonRocha}
    }
    {2019}
    {1606.03802}
    {Specialized Support Vector Machines for open-set recognition}
    {TSMCS}
    {MendesJunior2021}
  }{}%
  \ifthenelse{\equal{#1}{2011}}{%
    \proceedings{ref:MendesJunior2011}
    {
      \authorsymbol{PedroMendesJunior}
      \authorsymbol{JoseMariaNeves}
      \authorsymbol{AndreaTavares}
      \authorsymbol[]{DavidMenotti}
    }
    {2011}
    {http://dx.doi.org/10.1109/icsmc.2011.6084108}
    {Towards an automatic vehicle access control system: License plate location}
    {smc2011}
    {2916--2921}
    {}
    {MendesJunior2011}
  }{}%
  \ifthenelse{\equal{#1}{2019}}{%
    \journal{ref:MendesJunior2019}
    {
      \authorsymbol{PedroMendesJunior}
      \authorsymbol{LucaBondi}
      \authorsymbol{PaoloBestagini}
      \authorsymbol{StefanoTubaro}
      \authorsymbol[]{AndersonRocha}
    }
    {2019}
    {https://doi.org/10.1109/ACCESS.2019.2921436}
    {An in-depth study on open-set camera model identification}
    {a}
    {7}
    {}
    {180713--180726}
    {MendesJunior2019}
  }{}%
  \ifthenelse{\equal{#1}{2018b}}{%
    \thesis{ref:MendesJunior2018b}
    {2018}
    {http://repositorio.unicamp.br/handle/REPOSIP/333051}
    {Open-set recognition for different classifiers}
    {unicamp}
    {PhD}
    {MendesJunior2018b}
    {}
  }{}%
}

\urlstyle{same}


\title{Pedro Ribeiro Mendes Júnior}

\author{Postdoctoral Researcher\\%
  \\%
  \newtabhref{http://www.unicamp.br}{\glsfirst{unicamp}}\\%
  \newtabhref{https://www.ic.unicamp.br/en}{\glsfirst{ic}}\\%
  \newtabhref{http://www.recod.ic.unicamp.br}{\gls{recod}}\\%
  % \\%
  % \newtabhref{https://www.polimi.it}{\glsfirst{polimi}}\\%
  % \newtabhref{https://www.deib.polimi.it/eng/home-page}{\glsfirst{deib}}\\%
  % \newtabhref{http://ispg.deib.polimi.it/index.html}{\gls{ispg}}\\%
}
\date{}

\begin{document}

\maketitle

Currently, I am a postdoctoral researcher at the \gls{ic} of \gls{unicamp}.

For academic background, consider the following.
For a great part of 2019, I have been in a short-time postdoctoral research at \gls{polimi} in Italy investigating forensic problems in the context of open-set scenarios \ref{ref:MendesJunior2019} \ref{ref:MendesJunior2019b}.
I have obtained the master's degree \ref{ref:MendesJunior2014b} as well as doctor's degree \ref{ref:MendesJunior2018b} through \gls{unicamp} in Brazil investigating recognition in open-set scenarios \ref{ref:MendesJunior2017} \ref{ref:MendesJunior2016a}.
As extra works, I have some collaborations---also in machine learning---in medical problems \ref{ref:Pazinato2016} \ref{ref:Penatti2015}.
Through \gls{ufop-en} I had obtained the bachelor's degree in Computer Science \ref{ref:MendesJunior2012}.
During the bachelors, I had worked mainly with digital image processing in the specific problem of vehicle license plate location \ref{ref:MendesJunior2011} and I concluded my education studying the Haskell functional programming language for image processing \ref{ref:MendesJunior2012}.

\begin{flushright}
\newtabhref{http://lattes.cnpq.br/4535492803722330}{CV (Portuguese)}
\\
\newtabhref{https://scholar.google.com/citations?user=22PPnMAAAAAJ}{Google Scholar}
\\
\newtabhref{https://www.researchgate.net/profile/Pedro_Junior16}{ResearchGate}
\\
\newtabhref{https://recodbr.wordpress.com/team/students}{\gls{recod} page}
\\
\newtabhref{https://github.com/pedrormjunior}{Github page}
\\
\newtabhref{http://pedrormjunior.github.io/CV.pdf}{CV (PDF)}
\\
\newtabhref{https://orcid.org/0000-0001-8086-018X}{ORCID}
\end{flushright}


\section*{Reproducibility}
\label{sec:reproducibility}

Supplementary material for Chapter 7 of \ref{ref:MendesJunior2018b}: \href{OSNNet.html}{Page}
\\
Supplementary material for \ref{ref:MendesJunior2019}: \href{oscmi.html}{Page}
\\
Results and statistical analysis of \ref{ref:MendesJunior2016a}: \newtabhref{https://github.com/pedrormjunior/ssvm-results}{Github}
\\
Feature vectors employed in \ref{ref:MendesJunior2017}: \newtabhref{https://dx.doi.org/10.6084/m9.figshare.1097614}{FigShare}
\\
Proposed algorithm and baselines employed in \ref{ref:MendesJunior2011}: \newtabhref{https://github.com/pedrormjunior/vlpl}{Github}
\\
Datasets employed in \ref{ref:MendesJunior2011}: \href{dataset-VLPL.html}{Page}

\section*{Publications}

\subsection*{Journal articles}

\begin{enumerate}
\item\MendesJunior{2019}
\item\journal{ref:Neira2018}
  {
    \authorsymbol{ManuelNeira}
    \authorsymbol{PedroMendesJunior}
    \authorsymbol{AndersonRocha}
    \authorsymbol[]{RicardoTorres}
  }
  {2018}
  {https://doi.org/10.1109/ACCESS.2018.2824240}
  {Data-fusion techniques for open-set recognition problems}
  {a}
  {6}
  {}
  {21242--21265}
  {Neira2018}
\item\journal{ref:Werneck2018}
  {
    \authorsymbol{RafaelWerneck}
    \authorsymbol{WaldirAlmeida}
    \authorsymbol{BernardoStein}
    \authorsymbol{DanielPazinato}
    \authorsymbol{PedroMendesJunior}
    \authorsymbol{OtavioPenatti}
    \authorsymbol{AndersonRocha}
    \authorsymbol[]{RicardoTorres}
  }
  {2018}
  {https://doi.org/10.1016/j.future.2017.06.013}
  {Kuaa: A unified framework for design, deployment, execution, and recommendation of machine learning experiments}
  {fgcs}
  {78}
  {1}
  {59--76}
  {Werneck2018}
\item\journal{ref:MendesJunior2017}
  {
    \authorsymbol{PedroMendesJunior}
    \authorsymbol{RobertoSouza}
    \authorsymbol{RafaelWerneck}
    \authorsymbol{BernardoStein}
    \authorsymbol{DanielPazinato}
    \authorsymbol{WaldirAlmeida}
    \authorsymbol{OtavioPenatti}
    \authorsymbol{RicardoTorres}
    \authorsymbol[]{AndersonRocha}
  }
  {2017}
  {https://doi.org/10.1007/s10994-016-5610-8}
  {Nearest neighbors distance ratio open-set classifier}
  {ml}
  {106}
  {3}
  {359--386}
  {MendesJunior2017}
\item\journal{ref:Pazinato2016}
  {
    \authorsymbol{DanielPazinato}
    \authorsymbol{BernardoStein}
    \authorsymbol{WaldirAlmeida}
    \authorsymbol{RafaelWerneck}
    \authorsymbol{PedroMendesJunior}
    \authorsymbol{OtavioPenatti}
    \authorsymbol{RicardoTorres}
    \authorsymbol{FabioMenezes}
    \authorsymbol[]{AndersonRocha}
  }
  {2016}
  {http://dx.doi.org/10.1109/JBHI.2014.2386796}
  {Pixel-level tissue classification for ultrasound images}
  {jbhi}
  {20}
  {1}
  {256--267}
  {Pazinato2016}
\item\journal{ref:Penatti2015}
  {
    \authorsymbol{OtavioPenatti}
    \authorsymbol{RafaelWerneck}
    \authorsymbol{WaldirAlmeida}
    \authorsymbol{BernardoStein}
    \authorsymbol{DanielPazinato}
    \authorsymbol{PedroMendesJunior}
    \authorsymbol{RicardoTorres}
    \authorsymbol[]{AndersonRocha}
  }
  {2015}
  {http://dx.doi.org/10.1016/j.compbiomed.2015.08.004}
  {Mid-level image representations for real-time heart view plane classification of echocardiograms}
  {compbiomed}
  {66}
  {}
  {66--81}
  {Penatti2015}
  \speciallink{https://doi.org/10.1016/j.compbiomed.2016.05.001}{Honored papers 2015}
\end{enumerate}

\subsection*{Proceedings}

\begin{enumerate}[resume]
\item\proceedings{ref:MendesJunior2019b}
  {
    \authorsymbol{PedroMendesJunior}
    \authorsymbol{LucaBondi}
    \authorsymbol{PaoloBestagini}
    \authorsymbol{AndersonRocha}
    \authorsymbol[]{StefanoTubaro}
  }
  {2019}
  {https://doi.org/10.1109/ICIP.2019.8802981}
  {A PRNU-based method to expose video device compositions in open-set setups}
  {icip}
  {96--100}
  {}
  {MendesJunior2019b}
  \speciallink{https://cmsworkshops.com/ICIP2019/Papers/ViewPaper.asp?PaperNum=2462}{ICIP Session Details}
\item\proceedings{ref:Li2015}
  {
    \authorsymbol{LinLi}
    \authorsymbol{JavierMunoz}
    \authorsymbol{JurandyAlmeida}
    \authorsymbol{RodrigoCalumby}
    \authorsymbol{OtavioPenatti}
    \authorsymbol{IcaroDourado}
    \authorsymbol{KeillerNogueira}
    \authorsymbol{PedroMendesJunior}
    \authorsymbol{LuisPereira}
    \authorsymbol{DanielPedronette}
    \authorsymbol{JeferssonSantos}
    \authorsymbol{MarcosGoncalves}
    \authorsymbol[]{RicardoTorres}
  }
  {2015}
  {http://ceur-ws.org/Vol-1436/Paper49.pdf}
  {RECOD @ Placing Task of MediaEval 2015}
  {mediaeval2015}
  {1--3}
  {}
  {}
\item\proceedings{ref:Li2014}
  {
    \authorsymbol{LinLi}
    \authorsymbol{OtavioPenatti}
    \authorsymbol{JurandyAlmeida}
    \authorsymbol{GiovaniChiachia}
    \authorsymbol{RodrigoCalumby}
    \authorsymbol{PedroMendesJunior}
    \authorsymbol{DanielPedronette}
    \authorsymbol[]{RicardoTorres}
  }
  {2014}
  {http://ceur-ws.org/Vol-1263/mediaeval2014_submission_81.pdf}
  {Multimedia geocoding: The RECOD 2014 approach}
  {mediaeval2014}
  {1--2}
  {}
  {}
\item\MendesJunior{2011}
\item\proceedings{ref:MendesJunior2010}
  {
    \authorsymbol{PedroMendesJunior}
    \authorsymbol{JoseMariaNeves}
    \authorsymbol[]{DavidMenotti}
  }
  {2010}
  {PDFs/MendesJunior2010.pdf}
  {Em rumo a um sistema automático de controle de acesso de veículos automotivos: Localização de placas de identificação}
  {wuwsibgrapi}
  {1--6}
  {}
  {MendesJunior2010}
  \speciallink{http://www.inf.ufrgs.br/sibgrapi2010/index.php/call-for-participation/workshop-of-undergraduate-works/}{Accepted Contributions}
\end{enumerate}

\subsection*{Submitted papers}

\begin{enumerate}[resume]
\item\arxiv{ref:MendesJunior2016a}
  {
    \authorsymbol{PedroMendesJunior}
    \authorsymbol{TerranceBoult}
    \authorsymbol{JacquesWainer}
    \authorsymbol[]{AndersonRocha}
  }
  {2019}
  {https://arxiv.org/abs/}
  {1606.03802}
  {Specialized Support Vector Machines for open-set recognition}
  {}
  {MendesJunior2019a}
\end{enumerate}

\subsection*{Theses}

% #1: label
% #2: year
% #3: doi
% #4: title
% #5: institution
% #6: BibTeX
\begin{enumerate}[resume]
\item\MendesJunior{2018b}
  \speciallink{http://www.ic.unicamp.br/node/1365}{\gls{unicamp}'s news}
  \speciallink{https://recodbr.wordpress.com/2018/10/03/three-ph-d-thesis-defense-allan-luis-e-pedro/}{\gls{recod}'s news (with photos)}
\item\thesis{ref:MendesJunior2014b}
  {2014}
  {http://repositorio.unicamp.br/jspui/handle/REPOSIP/275530}
  {Open-Set Optimum-Path Forest classifier}
  {unicamp}
  {MSc}
  {MendesJunior2014b}
  {}
\item\thesis{ref:MendesJunior2012}
  {2012}
  {http://www.decom.ufop.br/menotti/monoII121/files/BCC391-121-vf-08.2.4114-PedroRibeiroMendesJunior.pdf}
  {Uso de paralelismo de dados em algoritmos de processamento de imagens utilizando Haskell}
  {ufop}
  {BCS}
  {MendesJunior2012}
  {true}
\end{enumerate}

\subsection*{Patents}

\begin{enumerate}[resume]
\item\patent{ref:MendesJunior2014a}
  {
    \authorsymbol{PedroMendesJunior}
    \authorsymbol{RobertoSouza}
    \authorsymbol{RafaelWerneck}
    \authorsymbol{BernardoStein}
    \authorsymbol{DanielPazinato}
    \authorsymbol{WaldirAlmeida}
    \authorsymbol{OtavioPenatti}
    \authorsymbol{RicardoTorres}
    \authorsymbol[]{AndersonRocha}
  }
  {2014}
  {https://patents.google.com/patent/US10133988B2}
  {Method for multiclass classification in open-set scenarios and uses thereof}
  {uspto}
  {us}
  {\href{https://patents.google.com/patent/US20160092790A1}{US20160092790A1}, \href{https://patents.google.com/patent/US10133988B2}{US10133988B2}}
  {\formatdate{2014}{11}{04}}
  {\formatdate{2016}{03}{31}}
  {\formatdate{2018}{11}{20}}
\item\patent{ref:MendesJunior2014}
  {
    \authorsymbol{PedroMendesJunior}
    \authorsymbol{RobertoSouza}
    \authorsymbol{RafaelWerneck}
    \authorsymbol{BernardoStein}
    \authorsymbol{DanielPazinato}
    \authorsymbol{WaldirAlmeida}
    \authorsymbol{OtavioPenatti}
    \authorsymbol{RicardoTorres}
    \authorsymbol[]{AndersonRocha}
  }
  {2014}
  {https://patents.google.com/patent/BR102014023780A2}%{https://worldwide.espacenet.com/publicationDetails/originalDocument?FT=D&date=20160517&DB=&locale=&CC=BR&NR=102014023780A2&KC=A2&ND=1}
  {Método para classificação multiclasse em cenários abertos e usos do mesmo}
  {inpi}
  {br}
  {\href{https://patents.google.com/patent/BR102014023780A2}{BR102014023780A2}}
  {\formatdate{2014}{09}{25}}
  {\formatdate{2016}{05}{17}}
  {}
\item\patent{ref:Rocha2014}
  {
    \authorsymbol{AndersonRocha}
    \authorsymbol{RicardoTorres}
    \authorsymbol{OtavioPenatti}
    \authorsymbol{PedroMendesJunior}
    \authorsymbol{DanielPazinato}
    \authorsymbol{WaldirAlmeida}
    \authorsymbol{RafaelWerneck}
    \authorsymbol[]{BernardoStein}
  }
  {2014}
  {https://patents.google.com/patent/BR102014011059A2}
  {Método para classificação automática de visões do coração a partir de ecocardiogramas}
  {inpi}
  {br}
  {\href{https://patents.google.com/patent/BR102014011059A2}{BR102014011059A2}}
  {\formatdate{2014}{05}{07}}
  {\formatdate{2018}{12}{04}}
  {}
\end{enumerate}

% \section*{News}

% \begin{enumerate}[resume]
%   \item \newtabhref{https://recodbr.wordpress.com/2016/06/02/recod-is-among-the-winners-of-unicamp-inventor-award-2016}{RECOD is among the winners of Unicamp Inventor Award 2016}
% \end{enumerate}

% -*- eval: (git-gutter-mode); -*-
\begin{footenv}
  \begin{footinfo}
    \glsfirst{PedroMendesJunior}\\
    \href{mailto:pedrormjunior@gmail.com}{pedrormjunior@gmail.com}\\
    \href{https://pedrormjunior.github.io/}{pedrormjunior.github.io}\\
    \href{tel:+55-19-3521-0340}{P +55 (19) 3521-0340}\\
    \newtabhref{http://www.unicamp.br}{\glsdesc{unicamp}}\\
    \newtabhref{https://www.ic.unicamp.br/en}{\glsdesc{ic}}\\
    \newtabhref{https://goo.gl/maps/sdLyWCsS4kL2}{%
      Av. Albert Einstein, 1251\\
      Cidade Universitária, 13083-852\\
      Campinas, São Paulo, Brazil
    }\\
  \end{footinfo}
\end{footenv}


\end{document}
